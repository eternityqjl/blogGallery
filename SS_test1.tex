\documentclass[UTF8]{ctexart}

\usepackage{graphicx}
\usepackage{amsmath}

\usepackage[colorlinks=true]{hyperref}

\usepackage{listings}
\lstset{
language=Matlab,
escapeinside=``, 
numbers=left,
numberstyle=\tiny,
breaklines=true, 
backgroundcolor=\color{lightgray!40!white},
frame=single,
framerule=0pt,
extendedchars=false, 
keywordstyle=\color{blue!70}\bfseries, 
basicstyle=\ttfamily,
commentstyle=\ttfamily\color{green!40!black}, 
showstringspaces=false}

\usepackage{xcolor}
%\usepackage{fontspec}
%\setmonofont{Consolas}

\usepackage{geometry}
\geometry{papersize={21cm,29.7cm}}
\geometry{left=2.5cm, right=2.5cm, top=2.5cm, bottom=2.5cm}
\usepackage{lmodern}

\usepackage{fancyhdr}
\pagestyle{fancy}
\rhead{信号与系统-实验报告}
\lhead{实验1}
\cfoot{\thepage}


\begin{document}


\begin{titlepage}

	\vspace*{0.5cm}

	\begin{figure}[h]
		\centering
		\includegraphics[width=0.9\linewidth]{logo}
	\end{figure}

	\vspace*{1.0cm}
	
	\begin{center}
		\fontsize{45pt}{0}\textbf{实验报告(本科)} \\
		\vspace*{1cm}
		\Huge{\textbf{——信号与系统课程}}
	\end{center}
	
	\vspace*{2.0cm}
	
	\begin{center}
		\LARGE 学\qquad号:\ \underline{\makebox[200pt]{2018141231122}} \\
		\LARGE 姓\qquad名:\ \underline{\makebox[200pt]{乔佳龙}} \\
		\LARGE 专\qquad业:\ \underline{\makebox[200pt]{自动化}} \\
		\LARGE 日\qquad期:\ \underline{\makebox[200pt]{2020年3月22日}} \\
		\LARGE 实验题目:\ \underline{\makebox[200pt]{线性时不变系统}} \\
		\LARGE 实验内容:\ \underline{\makebox[200pt][l]{信号的产生(sin, square,}}\\
		\underline{\makebox[296pt][l]{sinc, exp,sawtooth),波形的绘制和}} \\
		\underline{\makebox[296pt][l]{分析,卷积的计算。}} \\
%		\underline{\makebox[296pt][l]{}} \\
%		\underline{\makebox[296pt][l]{}} \\
%		\underline{\makebox[296pt][l]{}} \\
	\end{center}
\end{titlepage}

\newpage
\tableofcontents

\newpage
%1
\section{理论计算}
%1.1
\subsection[第3题(1)]{第3题(1)求\texorpdfstring{$x_1(t)=1.5e^{-2t}u(t)$}{x1},\texorpdfstring{$x_2(t)=cos(t2)[u(t)-u(t-2)]$}{x2}的卷积积分\texorpdfstring{$y_1(t)=x_1(t)*x_2(t)$}{y1}。 }
\begin{align}
y_1(t)
& =\int_{-\infty}^{+\infty}x_2(\tau)x_1(t-\tau)\mathrm{d}\tau \\
& =\int_{-\infty}^{+\infty}\cos(2\tau)[u(\tau)-u(\tau -2)]\times1.5\times e^{-2(t-\tau)}u(t-\tau)\mathrm{d}\tau \\
& =\int_{0}^{2}1.5\times \cos(2\tau)e^{-2(t-\tau)}u(t-\tau) \mathrm{d}\tau \\
& = \begin{cases}
	0,\ t\leq0 \\
	\int_{0}^{t}1.5\times \cos(2\tau)e^{-2(t-\tau)}\mathrm{d}\tau,\ 0 < t < 2 \\
	\int_{0}^{2}1.5\times \cos(2\tau)e^{-2(t-\tau)}\mathrm{d}\tau,\ t \geq2
	\end{cases} \\
& = \begin{cases}
	0,\ t\leq0 \\
	\frac{3}{8}[\sin(2t)+\cos(2t)-e^{-2t}], \ 0 < t < 2 \\
	\frac{3}{8}[(\sin(4)+\cos(4))e^{4-2t}-e^{-2t}], \ t \geq 2
	\end{cases}
\end{align}

%1.2
\subsection[第3题(2)]{第三题(2)求\texorpdfstring{$x_3[n]=(-\frac{2}{3})^n u[n-1]$}{x3},\texorpdfstring{$x_4[n]=(-1)^{n+1}u[n-1]-(-2)^{n-2}u[n-2]$}{x4}的卷积和\texorpdfstring{$y_2[n]
=x_3[n]*x_4[n]$}{y2}}
\begin{align}
y_2[n]
&=x_3[n]*x_4[n] \\
&=((-\frac{2}{3})^n u[n-1])*((-1)^{n+1}u[n-1]-(-2)^{n-2}u[n-2]) \\
&=\sum_{k=-\infty}^{+\infty}(-\frac{2}{3})^k u[k-1]((-1)^{n-k+1}u[n-k+1]-(-2)^{n-k-2}u[n-k-2]) \\
&=\sum_{k=1}^{+\infty}(-\frac{2}{3})^k ((-1)^{n-k+1}u[n-k+1]-(-2)^{n-k-2}u[n-k-2]) \\
&=
\begin{cases}
	0, \ n<0 \\
	\sum_{k=1}^{+\infty}(-\frac{2}{3})^k (-1)^{n-k+1}u[n-k+1], \ 0 \leq n \leq 2 \\
	\sum_{k=1}^{+\infty}(-\frac{2}{3})^k ((-1)^{n-k+1}u[n-k+1]-(-2)^{n-k-2}u[n-k-2]),\ n>2
\end{cases}
\end{align}
\paragraph{
所以,代入可求得以下结果:
}
\begin{align}
y_2[0]&=-\frac{2}{3} \\
y_2[1]&=\frac{10}{9} \\
y_2[2]&=-\frac{37}{27} \\
y_2[3]&=\frac{184}{81}
\end{align}



\newpage
%2
\section{仿真结果(程序、时域图、频域图)}
%2.1
\subsection{第1题}
\subsubsection{程序}
\begin{lstlisting}
%range of independent variable t
t=-4:0.001:4;

%sine function
subplot(3,2,1)
x1=sin(pi.*t);
plot(t,x1,'Color','\#0072BD','LineWidth',1)
title('sin function: $$ x_1=\sin(\pi t) $$','Interpreter','latex')
xlim([-5,5])
xticks(-5:1:5)
yticks(-1:0.5:1)
xlabel('t')
ylabel('$$x_1$$','Interpreter','latex')
grid on

%sinc function
subplot(3,2,2)
x2=sinc(pi*t);
plot(t,x2,'Color','\#0072BD','LineWidth',1)
title('sinc function: $$ x_2=\mathrm{sinc}(\pi t) $$','Interpreter','latex')
xlim([-5,5])
xticks(-5:1:5)
grid on

%imaginary exponent signal
subplot(3,2,3)
x3=exp(-1i*pi*t);
plot(t,x3,'Color','\#77AC30','LineWidth',1)
title('$$ x_3=e^{-i \pi t} $$','Interpreter','latex')
xlim([-5,5])
xticks(-5:1:5)
grid on

%real exponent signal
subplot(3,2,4)
x4=exp(-t);
plot(t,x4,'Color','\#77AC30','LineWidth',1)
title('$$x_4=e^{-t}$$','Interpreter','latex')
xlim([-5,5])
xticks(-5:1:5)
grid on

%square wave signal(T: 2, duty circle: 30%, amplitude: +1.5/-1.5)
subplot(3,2,5)
x5=1.5*square(2*t,30);
plot(t,x5,'Color','\#A2142F','LineWidth',1)
title('Square wave signal(T = 2, duty circle = 30%, amplidute = +1.5/-1.5)')
xlim([-5,5])
xticks(-5:1:5)
grid on

%Triangular wave (T: 1, amplitude: 1)
subplot(3,2,6)
x6=sawtooth(2*pi*t,0.8);
plot(t,x6,'Color','\#A2142F','LineWidth',1)
title('Triangular wave (T = 1, amplitude = 1)')
xlim([-5,5])
xticks(-5:1:5)
grid on
\end{lstlisting}

\subsubsection{时域图}
\begin{figure}[h]
\centering
\includegraphics[width=1\linewidth]{signal1}
\end{figure}

%2.2
\subsection{第2题(1)}
\subsubsection{程序}
\begin{lstlisting}
clc
clear
close all

%range of independent variable t
t=-10:0.001:10;

%x(t)
x1=sinc((1/pi)*t);
plot(t,x1,'Color','\#0072BD','LineWidth',1)
xlim([-12,12])
ylim([-0.4,1.2])
xlabel('t')
grid on
title('Comparison of $$x(t)\ and \ x(-t+2)$$','Interpreter','latex')

hold on

%x(-t+2)
x2=sinc((1/pi)*(-t+2));
plot(t,x2,'Color','\#D95319','LineWidth',1)

%legend
legend('$$\mathrm{sinc}(\frac{t}{\pi})$$','$$\mathrm{sinc}(\frac{-t+2}{\pi})$$','Interpreter','latex')
\end{lstlisting}
\subsubsection{时域图}
\begin{figure}[h]
\centering
\includegraphics[width=1\linewidth]{signal2_1}
\end{figure}

%2.3
\subsection{第2题(2)}
\subsubsection{程序}
\begin{lstlisting}
clc
clear
close all
%2-2

%x_N[n]
subplot(2,1,1)
x1=linspace(-5,9,15);
y1=[0.5,1.5,2,1,0, 0.5,1.5,2,1,0, 0.5,1.5,2,1,0,];
stem(x1,y1,'filled','LineWidth',1)
title('$$X_N[n]; \ N=5$$','Interpreter','latex')
xlim([-5,10])
ylim([0,2.5])
xticks(-5:1:10)
grid on

%x_N[-n]
subplot(2,1,2)
x2=linspace(-4,10,15);
y2=[0,1,2,1.5,0.5, 0,1,2,1.5,0.5, 0,1,2,1.5,0.5,];
stem(x2,y2,'filled','LineWidth',1,'MarkerEdgeColor','r')
title('$$X_N[-n]; \ N=5$$','Interpreter','latex')
xlim([-5,10])
ylim([0,2.5])
xticks(-5:1:10)
grid on
\end{lstlisting}
\subsubsection{时域图}
\newpage
\begin{figure}[h]
\centering
\includegraphics[width=1\linewidth]{signal2_2}
\end{figure}

%2.4
\subsection{第3题}
\subsubsection{程序}
\begin{lstlisting}
clc
clear
close all

%convolution y1(t)
%range of independent variable t0 of x1(t) and x2(t)
t0 = -5:0.1:5;
%x1(t)
subplot(2,3,1)
x1 = 1.5*exp(-2*t0).*stepfun(t0,0);
plot(t0,x1,'LineWidth',1,'Color','\#0072BD')
title('$$x_1(t)=1.5 e^{-2t}u(t)$$','Interpreter','latex')
xlabel('$$t$$','Interpreter','latex')
xticks(-5:1:5)
grid on

%x2(t)
subplot(2,3,2)
x2 = cos(2*t0).*(stepfun(t0,0)-stepfun(t0,2));
plot(t0,x2,'LineWidth',1,'Color','\#7E2F8E')
title('$$x_2(t)=\cos(2t)[u(t)-u(t-2)]$$','Interpreter','latex')
xlabel('$$t$$','Interpreter','latex')
xticks(-5:1:5)
grid on

%y1(t)
%range of independent variable t1 of y1(t)
t1 = -10:0.1:10;

subplot(2,3,3)
y1 = conv(x1,x2);
plot(t1,y1,'LineWidth',1,'Color','\#A2142F')
title('$$y_1(t)=x_1(t)*x_2(t)$$','Interpreter','latex')
xlabel('$$t$$','Interpreter','latex')
xticks(-10:1:10)
grid on
hold on


%convolution y2[n]
%range of independent variable n0 of x3[n] and x4[n]
n0=-4:1:6;

%x3[n]
subplot(2,3,4)
x3 = ((-2/3).^n0).*stepfun(n0,1);
stem(n0,x3,'filled','LineWidth',1,'Color','\#0072BD')
title('$$x_3[n]=(-\frac{2}{3})^n u[n-1]$$','Interpreter','latex')
xlabel('$$n$$','Interpreter','latex')
xlim([-5,7])
xticks(-5:1:7)
grid on

%x4[n]
subplot(2,3,5)
x4 = (-1).^(n0+1).*stepfun(n0,-1)-((-2).^(n0-2)).*stepfun(n0,2);
stem(n0,x4,'filled','LineWidth',1,'Color','\#7E2F8E')
title('$$x_4[n]=(-1)^{n+1}u[n-1]-(-2)^{n-2}u[n-2]$$','Interpreter','latex')
xlabel('$$n$$','Interpreter','latex')
xlim([-5,7])
xticks(-5:1:7)
grid on

%y2[n]
%range of independent variable n1 of y2[n]
ly2 = 2.*n0(1);
hy2 = 2.*n0(end);
n1 = ly2:1:hy2;

subplot(2,3,6)
y2 = conv(x3,x4);
stem(n1,y2,'filled','LineWidth',1,'Color','\#A2142F')
title('$$y_2[n]=x_3[n]*x_4[n]$$','Interpreter','latex')
xlabel('$$n$$','Interpreter','latex')
xlim([-9,13])
xticks(-9:1:13)
grid on
\end{lstlisting}
\subsubsection{时序图}
\begin{figure}[h]
\centering
\includegraphics[width=1\linewidth]{signal3}
\end{figure}


\newpage
%3
\section{结果分析及问题回答}
%3.1
\subsection{第2题(1)}
\paragraph{
二者的相位相差2。因为$sinc(t)$函数是一个偶函数,将其沿y轴变换为$sinc(-t)$时波形不变,与原函数保持一致,所以二者只是存在大小为2的相位差。
}
\subsection{第2题(2)}
\paragraph{
两图像的周期和自变量的取值范围时相同的,区别是每个周期内各自变量处的取值顺序不同,将第一幅图中每个周期内的图像都进行反转即可得到第二幅图。这也及将第一幅图沿y轴变换后得到的结果。
}
\subsection{第3题(1)}
\paragraph{
Matlab中的conv(a,b)函数是通过两个向量a、b计算卷积的,因此如果要计算连续函数的卷积,就需要将函数的自变量区间划分为很小的区间,这样就能计算得到连续函数的卷积。但是,我发现一些问题,对自变量取不同的小区间宽度时得到的结果并不相同,如下所示,这个问题还未解决,存在疑惑。
}
\subsection{第3题(2)}
\paragraph{
从第一部分的理论计算中我们可以得到以下函数值:
}
\begin{align}
y_2[0]&=-\frac{2}{3} \\
y_2[1]&=\frac{10}{9} \\
y_2[2]&=-\frac{37}{27} \\
y_2[3]&=\frac{184}{81}
\end{align}
\paragraph{
而使用用数学工具Matlab仿真得到的结果如图上图所示,二者相等,所以仿真结果正确。
}
\begin{figure}[h]
\centering
\includegraphics[width=1\linewidth]{signal3_2}
\end{figure}


\newpage
%4
\section{调试分析及总结}
\subsection{程序调试分析}

\subsubsection{问题1:conv(a,b)函数的使用}
最初在使用conv函数的结果y与自变量t作图时经常会出现以下的报错:
\begin{lstlisting}
Error using plot
Vectors must be the same length.
\end{lstlisting}
通过查阅Matlab文档发现,使用conv(a,b)函数求得的卷积结果的向量长度发生了变化,长度变为了a的长度加上b的长度再减1,所以在作卷积结果图时需要对自变量的范围做出更改,将原函数自变量的最小值和最大值都乘以2即可。

\subsubsection{问题2:m程序编写的顺序}
最初写作图的plot函数和相关参数设置语句时没有注意程序的先后顺序,以下面的程序为例:
\begin{lstlisting}
subplot(2,3,1)
x1 = 1.5*exp(-2*t0).*stepfun(t0,0);
title('$$x_1(t)=1.5 e^{-2t}u(t)$$','Interpreter','latex')
xlabel('$$t$$','Interpreter','latex')
xticks(-5:1:5)
grid on
plot(t0,x1,'LineWidth',1,'Color','\#0072BD')
\end{lstlisting}
该程序中将title、xlabel、grid on等控制图的参数的语句写在了plot函数的前面,导致这些参数设置无法在图中显示。解决方法时将这些语句放在plot函数后即可。以后写程序过程中需要注意顺序,以免再发生这种问题。

\subsubsection{
其他:第一次调试LaTex程序中发现的问题
}
这部分内容我写在了自己的博客中。


\subsection{总结}
\paragraph{
\quad 这是信号与系统课程的第一次实验,主要是来学习Matlab的基本操作内容,以及计算线性时不变系统中的卷积,通过理论计算和使用软件仿真来实现。实验中主要存在的问题有:\\
\\
1.对于Matlab软件的基本操作和一些基本的函数内容不熟悉,导致编写程序的过程中经常出错。需要多查询Matlab的官方文档,并且通过实验逐渐熟悉常用的内容。 \\
\\
2.实验顺序的问题。这次实验我没有先进行理论计算再进行软件仿真,导致刚开始并没有发现隐含的一些问题,最后需要反反复复地重复。所以在今后的实验中在仿真前要先进行理论计算,做到心中有数,对仿真的结果有一个预期,这样才能达到更好的效果。
}
\end{document}